\documentclass[12pt,a4paper]{report}

% Packages
\usepackage[utf8]{inputenc}
\usepackage[english]{babel}
\usepackage{graphicx}
\usepackage{amsmath}
\usepackage{amssymb}
\usepackage{hyperref}
\usepackage{cite}
\usepackage{algorithm}
\usepackage{algorithmic}
\usepackage{caption}
\usepackage{subcaption}
\usepackage{booktabs}
\usepackage{multirow}
\usepackage{xcolor}
\usepackage{listings}
\usepackage{geometry}

% Page geometry
\geometry{
    left=1.5in,
    right=1in,
    top=1in,
    bottom=1in
}

% Hyperref setup
\hypersetup{
    colorlinks=true,
    linkcolor=blue,
    filecolor=magenta,      
    urlcolor=cyan,
    citecolor=blue,
}

% Code listing setup
\lstset{
    basicstyle=\ttfamily\small,
    breaklines=true,
    frame=single,
    numbers=left,
    numberstyle=\tiny\color{gray},
}

% Title page information
\title{
    \textbf{Explainable AI for Docker Container Malware Detection\
    Using Vision Transformers}\
    \vspace{1cm}
    \large MTech Thesis\
    Indian Institute of Information Technology Guwahati
}
\author{
    [Your Name]\
    Roll No: [Your Roll Number]\
    \vspace{1cm}
    Supervisor: Dr. Ferdous Ahmed Barbhuiya
}
\date{February 2026}

\begin{document}

% Title Page
\maketitle

% Abstract
\chapter*{Abstract}
\addcontentsline{toc}{chapter}{Abstract}

This thesis investigates the application of explainable artificial intelligence (XAI) 
techniques to Docker container malware detection using vision transformers. Container 
security is a critical concern in modern cloud computing, with malware often modifying 
only a tiny fraction of the container filesystem, making detection challenging.

We extend the image-based malware detection approach by Nousias et al. (2025) by 
introducing explainability through GradCAM, SHAP, and LIME. We compare convolutional 
neural networks (CNNs) with vision transformer architectures (Swin, ViT) and evaluate 
which XAI methods best identify actual malware regions using ground-truth masks from 
the COSOCO dataset.

Furthermore, we propose a novel Hilbert curve inverse mapping technique that traces 
model attention back to specific byte offsets and files in the container filesystem, 
enabling forensic attribution. Our results show that [results to be filled] and 
demonstrate the practical utility of XAI for security analysts.

\textbf{Keywords:} Container Security, Malware Detection, Explainable AI, Vision Transformers, 
GradCAM, Docker, Deep Learning

% Table of Contents
\tableofcontents
\listoffigures
\listoftables

% Chapters
\chapter{Introduction}

\section{Background}

% TODO: Write about container security, malware threats, importance of detection

\section{Motivation}

% TODO: Explain why explainability is important, limitations of black-box models

\section{Problem Statement}

% TODO: Clearly state the research problem

\section{Research Questions}

\begin{enumerate}
    \item How do CNN and Vision Transformer architectures compare for container malware detection?
    \item Which XAI technique (GradCAM, SHAP, LIME) best identifies actual malware regions?
    \item Can we trace model decisions back to specific files in the container filesystem?
    \item Do transformers attend to more forensically relevant regions than CNNs?
\end{enumerate}

\section{Contributions}

% TODO: List the novel contributions of this thesis

\section{Thesis Organization}

% TODO: Briefly describe the structure of remaining chapters

\chapter{Literature Review}

\section{Container Security and Threats}

% TODO: Review container security literature

\section{Image-Based Malware Detection}

% TODO: Discuss image-based approaches (Nataraj et al., Nousias et al.)

\section{Deep Learning for Malware Analysis}

\subsection{Convolutional Neural Networks}

% TODO: Review CNN architectures used in malware detection

\subsection{Vision Transformers}

% TODO: Review ViT, Swin, and their applications to security

\section{Explainable AI Methods}

\subsection{GradCAM and Variants}

% TODO: Review GradCAM, HiResCAM, GradCAM++

\subsection{SHAP}

% TODO: Review SHAP and its applications

\subsection{LIME}

% TODO: Review LIME

\section{Related Work Summary}

% TODO: Summarize key findings from literature

\section{Research Gap}

% TODO: Clearly identify the gap this thesis addresses

\chapter{Methodology}

\section{Problem Formulation}

% TODO: Formally define the problem

\section{Overall Framework}

% TODO: Include methodology flowchart

\section{Data Pipeline}

\subsection{COSOCO Dataset}

% TODO: Describe dataset characteristics

\subsection{Patch Extraction}

% TODO: Explain patch extraction process

\subsection{Data Augmentation}

% TODO: Describe augmentation strategies

\section{Model Architectures}

\subsection{ResNet18 Baseline}

% TODO: Describe ResNet18 architecture and modifications

\subsection{Swin Transformer}

% TODO: Describe Swin architecture

\subsection{Vision Transformer}

% TODO: Describe ViT architecture

\subsection{Multiple Instance Learning}

% TODO: Explain MIL aggregation strategies

\section{XAI Methods}

\subsection{GradCAM}

% TODO: Explain GradCAM methodology

\subsection{HiResCAM}

% TODO: Explain HiResCAM improvements

\subsection{SHAP}

% TODO: Explain SHAP methodology

\subsection{LIME}

% TODO: Explain LIME methodology

\section{Hilbert Inverse Mapping}

\subsection{Mathematical Foundation}

% TODO: Explain Hilbert curve mathematics

\subsection{Coordinate to Byte Mapping}

% TODO: Explain inverse mapping algorithm

\subsection{Forensic File Attribution}

% TODO: Explain forensic analysis pipeline

\section{Evaluation Metrics}

% TODO: Define all evaluation metrics (classification + XAI)

\chapter{Implementation}

\section{Experimental Setup}

% TODO: Describe hardware, software, compute resources

\section{Dataset Preparation}

% TODO: Describe data splits, preprocessing steps

\section{Training Configuration}

% TODO: Detail hyperparameters, optimization settings

\section{XAI Implementation Details}

% TODO: Describe implementation specifics for each XAI method

\section{Hilbert Mapping Implementation}

% TODO: Describe implementation of inverse mapping

\section{Code Architecture}

% TODO: Briefly describe software organization

\chapter{Results and Analysis}

\section{Classification Results}

\subsection{Model Comparison}

% TODO: Present accuracy, F1, precision, recall for all models

\subsection{Confusion Matrices}

% TODO: Show confusion matrices

\subsection{ROC Analysis}

% TODO: Present ROC curves

\section{XAI Evaluation}

\subsection{IoU Analysis}

% TODO: Present IoU scores for different XAI methods

\subsection{Pointing Game Results}

% TODO: Present pointing game accuracy

\subsection{Faithfulness Evaluation}

% TODO: Present faithfulness metrics

\subsection{Visual Comparison}

% TODO: Show side-by-side visualizations

\section{Hilbert Inverse Mapping Validation}

% TODO: Validate inverse mapping accuracy

\section{Forensic Case Studies}

\subsection{Case Study 1: Mirai Detection}

% TODO: Detailed analysis of Mirai example

\subsection{Case Study 2: CoinMiner Detection}

% TODO: Detailed analysis of CoinMiner example

\subsection{Case Study 3: False Positive Analysis}

% TODO: Analyze a false positive case

\section{CNN vs ViT Discussion}

% TODO: Comprehensive comparison of CNN and ViT approaches

\section{Discussion}

% TODO: Interpret results, discuss implications

\chapter{Conclusion}

\section{Summary of Contributions}

% TODO: Summarize what was achieved

\section{Key Findings}

% TODO: Highlight main findings

\section{Limitations}

% TODO: Discuss limitations of the approach

\section{Future Work}

% TODO: Suggest future research directions

\section{Concluding Remarks}

% TODO: Final thoughts on the work


% Bibliography
\bibliographystyle{IEEEtran}
\bibliography{references}

% Appendices (optional)
% \appendix
% \chapter{Code Snippets}
% \chapter{Additional Results}

\end{document}
